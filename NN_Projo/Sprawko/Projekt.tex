\documentclass[12pt]{article}
\usepackage[utf8]{inputenc}
\usepackage[T1]{fontenc}
\usepackage{pdflscape} 
\usepackage{lmodern}
\usepackage[a4paper,bindingoffset=0.2in,%
            left=0.5in,right=0.5in,top=0.5in,bottom=1in,%
            footskip=.25in]{geometry}
\usepackage[colorlinks=true, linkcolor=Black, urlcolor=Blue]{hyperref}
\usepackage{graphicx}
\usepackage{subcaption}
\usepackage{listings}
\usepackage{color}
\usepackage[table]{xcolor}
\definecolor{lightgray}{gray}{0.9}

\definecolor{codegreen}{rgb}{0,0.6,0}
\definecolor{codegray}{rgb}{0.5,0.5,0.5}
\definecolor{codepurple}{rgb}{0.58,0,0.82}
\definecolor{backcolour}{rgb}{0.95,0.95,0.92}

\lstdefinestyle{mystyle}{
	backgroundcolor=\color{backcolour},   
	commentstyle=\color{codegreen},
	keywordstyle=\color{magenta},
	numberstyle=\tiny\color{codegray},
	stringstyle=\color{codepurple},
	basicstyle=\ttfamily\footnotesize,
	breakatwhitespace=false,         
	breaklines=true,                 
	captionpos=b,                    
	keepspaces=true,                 
	numbers=left,                    
	numbersep=5pt,                  
	showspaces=false,                
	showstringspaces=false,
	showtabs=false,                  
	tabsize=2
}


\begin{document}
\title{Projekt - KMeans\\
\large Sebastian Michoń 136770, Marcin Zatorski 136834\\
\large grupe L5}
\date{\vspace{-10ex}}
\maketitle

\section{Preprocessing}
\begin{enumerate}
	\item Kolumny 'name' i 'mfr' nie były używane w przetwarzaniu, kolumna 'type' została sprowadzona do zmiennej binarnej.
	\item Atrybuty zostały znormalizowane (sprowadzone do przedziału <0;1> w standardowy sposób, formułą \(\frac{x-min}{max-min}\))
	\item Z pozostałych atrybutów nie były przetwarzane "weight", "shelf", "cups" i "ratings" - jeśli zadanie dotyczyło produktów podobnych ze względu na wartości odżywcze, to te atrybuty są prawdopodobnie zbędne.
\end{enumerate}

\section{Informacje wyróżniające powstałe grupy}
\begin{enumerate}
	\item W klastrach było kolejno: 14, 25 i 38 obserwacji
	
	\item Obserwacje zawierające rodzynki (raisin w nazwie) były prawie wyłącznie w 3. grupie (9 z 11 obserwacji) podobnie jak musli (3 obserwacje) i miód (5 obserwacji). Kukurydza występowała głównie w 2. grupie (3 z 4 obserwacje), podobnie jak winogrona (2 obserwacje) i nasiona (4 z 5 obserwacji). W 1. grupie było 6 obserwacji z pszenicą w nazwie (były też 3 w 2. klastrze i 2 w 3. klastrze).
	
	\item Obserwacje z 1. klastra posiadały średnio najwyższe oceny (minimum wyższe niż maksimum trzeciego klastra), zawierały najmniej witamin, sodu i tłuszczu.
	
	\item W drugim klastrze obserwacje zawierały średnio najwięcej węglowodanów, sodu i witamin.
	
	\item W trzecim klastrze obserwacje zawierały średnio najwięcej cukru, tłuszczu i kalorii (najniższa wartość kalorii była większa niż najwyższa wartość dla 1. grupy). Ponadto miały najniższe oceny.
	
	\item Kilka składników odżywczych rozkładało się równomiernie między grupami - potas, białko, do pewnego stopnia błonnik.
\end{enumerate}

\section {Tezy dalsze}
\begin{enumerate}
	\item Zależnie od wybranych początkowych centroidów powstałe grupy są różne, przy czym pewne obserwacje zawsze występują razem.
	\item Rezultaty własnego kMeansa były podobne do rezultatów tego samego algorytmu ze scikit-learn.
\end{enumerate}

\end{document}
